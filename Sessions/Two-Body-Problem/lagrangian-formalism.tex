Considere un sistema de dos partículas $m_1$ y $m_2$ de la misma clase, con fuerzas centrales mutuas, cuyas posiciones están dadas en términos de los vectores $\vec{r}_1$ y $\vec{r}_2$. En primera, las fuerzas que actúan sobre cada partícula será función del vector distancia entre ellas definido por $\vec{r}=\vec{r}_2-\vec{r}_1$, por ende asociamos un potencial en función de la magnitud $r$ de forma $U(r,\dot{r},...)$, así nuestro sistema de 6 grados de libertad estará dado por el lagrangiano:
	
	\[
	L=T-U=\frac{1}{2}m_1\dot{r}_1^2+\frac{1}{2}m_2\dot{r}_2^2 - U(r,\dot{r},...)
	\]

	Ahora conviene escribir $\vec{r}_1$ y $\vec{r}_2$ en términos del centro de masa $\vec{R}$ y el vector $\vec{r}$, efectuando una transformación en las ecuaciones de Lagrange que preserva su forma, reescribimos entonces los vectores de la siguiente forma:
	
	\[
	\begin{array}{c}
		\vec{r}_1=\vec{R}+\vec{r'}_1\\\vec{r}_2=\vec{R}+\vec{r'}_2
	\end{array}
	\quad\quad\text{con}\quad\quad \vec{R}=\frac{m_1\vec{r}_1+m_2\vec{r}_2}{m_1+m_2}\quad \vec{r'}_1=-\frac{m_2}{m_1+m_2}\vec{r}\quad\vec{r'}_2=\frac{m_1}{m_1+m_2}\vec{r}
	\]
	
	Es necesario transformar el lagrangiano en función de $\vec{R}$ y $\vec{r}$ usando las ecuaciones de transformación previamente escritas. Notemos que, en el sistema primado donde se definen $\vec{r'}_1$ y $\vec{r'}_2$ (vectores que van del centro de masas hasta la partícula), el centro de masas coincide con el vector nulo:
	
	\[
	L=\frac{1}{2}m_1(\dot{R}^2+\dot{r'}^{2}_1)+\frac{1}{2}m_2(\dot{R}^2+\dot{r'}^{2}_2)+\dot{\vec{R}}\cdot\frac{d}{dt}(m_1\vec{r'}_1+m_2\vec{r'}_2) - U(r,\dot{r},...)
	\]
	
	Así el tercer termino es nulo y el lagrangiano toma la forma, en la que es necesario aplicar álgebra para dejarlo totalmente en función de $\vec{r}$:
	
	\[
	L=\frac{1}{2}(m_1+m_2)\dot{R}^{2}+\frac{1}{2}(m_1\dot{r'}^{2}_1+m_2\dot{r'}^{2}_2)-U(r,\dot{r},...)
	\]
	
	\[
	\dot{r'}^{2}_1=\frac{m_2^{2}}{(m_1+m_2)^{2}}\dot{r}^{2}\quad\dot{r'}^{2}_2=\frac{m_1^{2}}{(m_1+m_2)^{2}}\dot{r}^{2}\quad\quad m_1\dot{r'}^{2}_1+m_2\dot{r'}^{2}_2=\frac{m_1m_2}{m_1+m_2}\left[\frac{m_2\dot{r}^2+m_1\dot{r}^2}{m_1+m_2}\right]
	\]
	
	\[
	L=\frac{1}{2}(m_1+m_2)\dot{R}^{2}+\frac{1}{2}\frac{m_1m_2}{m_1+m_2}\dot{r}^2-U(r,\dot{r},...)
	\]
	
	Podemos notar que $R$ es cíclica, es decir, la lagrangiana no contiene $R$ (no tomar como definición) y por ende tendrá una cantidad de movimiento que se conserva, lo restante define un sistema de una sola partícula (con 3 grados de libertad) de masa reducida $\mu$ cuya posición esta definida por $\vec{r}$:
	
	\[
	L=\frac{1}{2}\frac{m_1m_2}{m_1+m_2}\dot{r}^2-U(r,\dot{r},...)=\frac{1}{2}\mu\dot{r}^2-U(r,\dot{r},...)
	\]
	
	Consideraremos ahora fuerzas conservativas, es decir $U=U(r)$, y la simetría esférica del sistema, note que se pueden hacer rotaciones alrededor de la esfera que define el vector $r$ sin alterar el lagrangiano del sistema, es decir, se conserva el momentum angular total:
	
	\[
	\vec{L}=\vec{r}\times\vec{p}\quad\quad \vec{L}(l,\theta,\phi)
	\]
	
	De aquí se sigue que $\vec{r}$ esta dado sobre el plano que define $\vec{L}$, si se fija $\theta$ y $\phi$ de $\vec{L}$ el numero de grados de libertad disminuye a 2 y contamos así con dos cantidades conservadas para solucionar el problema, $E$ y $l$.
